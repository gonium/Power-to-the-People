%/* vim: setformat tex
%*/

\section{Umsetzung}\label{sec:umsetzung}

Hier besteht die Gefahr, das Feld komplett den kommerziellen Anbietern
zu überlassen. Dies ist nicht unbedingt im Sinne der privaten
Verbraucher. Gleichzeitig werden auch kleinere Stadtwerke im Wettbewerb
übervorteilt, wenn große Stromanbieter ihre Infrastrukturen auf ihren
eigenen Gewinn hin optimieren. 

Wir gehen daher einen anderen Weg und bauen auf ein Ökosystem von freien
Lösungen. Dabei favorisieren wir Open-Source Projekte. Bei der Hardware
setzen wir auf Consumer-Geräte, die einen hohen Verbreitungsgrad haben.
Ein möglichst großer Anteil der Funktionen soll in Software
implementiert werden, da diese quasi ohne weitere Kosten vervielfältigt
werden kann.

Home Integration:
\begin{enumerate}
  \item Bus System: Schwierig, da im Moment kaum nachrüstbare Lösungen
	verfügbar sind. WLAN vs. digitalStrom vs. Pending?
  \item Integration der einzelnen Geräte: Möglixhst über TCP/IP
	ansprechbar (WLAN/Ethernet). Die einzelnen Funktionen müssen über
	REST ansprechbar sein, einzelne Geräte sollen über DNSSD/Bonjour
	auffindbar sein, jedes Gerät hat eine eindeutige, persistente ID.
\end{enumerate}

Dabei beachten: 
\begin{enumerate}
  \item Akzeptanz der Technologie bei den Anwendern: Diese müssen
	jederzeit in der Lage sein, die vorgeschlagenen Schedules
	abzulehnen. Visualisierung des Stromverbrauchs ist ebenso wichtig,
	bzw. auch eine Studie, welche Visualisierungen denn Sinn machen.
  \item Offenheit: Jederzeit soll eine Heiminstallation durch weitere
	Geräte erweiterbar sein. Die Anbindung von bislang nicht
	unterstützten Geräten soll durch kleine Adapter möglich sein.
\end{enumerate}

TODO: Architektur von mySmartGrid beschreiben.
