%/* vim: setf tex
%*/

\section{Demand Side Management}\label{sec:demandside}
Natürlich ist es erstrebenswert, absolut gesehen weniger Strom zu
verbrauchen\footnote{Die Autoren sind sich darüber im klaren, dass Energie
nicht verbraucht werden kann --- wir folgen jedoch dem allgemeinen
Sprachgebrauch.}. Allerdings ist dies nicht das Ziel des Demand-Side
Managements. Stattdessen wird hier versucht, den Verbrauch an die
Erzeugung von Strom anzupassen. Dieses Verfahren ist seit langem in der
Industrie üblich --- großen Verbrauchern aus der Industrie werden
vergünstigte Preise eingeräumt, wenn dafür im Falle von Netzengpässen
der Stromverbraucher abgeschaltet werden können. 

Dazu teilt der Netzbetreiber dem Großverbraucher auf kurze Sicht (15
Minuten) mit, dass einige Aggregate abzuschalten sind. Daraufhin werden
Spitzen im Strombedarf gekappt.

TODO: Verschiedene Szenarien, siehe eEnergy-Studie

Das gleiche Grundprinzip kann natürlich auch im Haushalt umgesetzt
werden. Hier muss es jedoch das Ziel sein, eine kostengünstige Umsetzung
ohne Komforteinbußen für den Verbraucher zu realisieren. Dazu spielt
natürlich auch der Preis eine Rolle.

