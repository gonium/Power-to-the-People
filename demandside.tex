%/* vim: setformat tex
%*/

\section{Demand Side Management}\label{sec:demandside}
Natürlich ist es erstrebenswert, absolut gesehen weniger Strom zu
verbrauchen\footnote{Die Autoren sind sich darüber im klaren, dass Energie
nicht verbraucht werden kann --- wir folgen jedoch dem allgemeinen
Sprachgebrauch.}. Allerdings ist dies nicht das Ziel des Demand-Side
Managements. Stattdessen wird hier versucht, den Verbrauch an die
Erzeugung von Strom anzupassen. Dieses Verfahren ist seit langem in der
Industrie üblich --- großen Verbrauchern aus der Industrie werden
vergünstigte Preise eingeräumt, wenn dafür im Falle von Netzengpässen
der Stromverbraucher abgeschaltet werden können. 

Dazu teilt der Netzbetreiber dem Großverbraucher auf kurze Sicht (15
Minuten) mit, dass einige Aggregate abzuschalten sind. Daraufhin werden
Spitzen im Strombedarf gekappt.

TODO: Verschiedene Szenarien, siehe eEnergy-Studie

Das gleiche Grundprinzip kann natürlich auch im Haushalt umgesetzt
werden. Hier muss es jedoch das Ziel sein, eine kostengünstige Umsetzung
ohne Komforteinbußen für den Verbraucher zu realisieren. Dazu spielt
natürlich auch der Preis eine Rolle. Zudem ist der verschiebbare
Stromverbrauch eines einzelnen Verbrauchers nicht groß im Vergleich zu
der benötigten Ausgleichsenergie. Allerdings kann durch die Kombination
einer hinreichend großen Anzahl an Haushalten ein signifikanter
Verbrauch verschoben werden. Diese Ansammlung von Haushalten und
natürlich auch Gewerbebetrieben kann als eine Einheit angesprochen
werden, wenn sie zu einem \emph{virtuellen Verbraucher}
zusammengeschlossen werden.

TODO: Kommerzielles Modell. Wie kann Demand-Side Management im
Privathaushalt nachhaltig werden? Angedacht ist der Verkauf der
Ausgleichsenergie an den Netzbetreiber. Dieser müsste nicht zu hohen
Preisen Strom kaufen bzw. verkaufen. Diese Marge kann sich der
Netzbetreiber mit den Privathaushalten teilen. Dzu gründen die
Privathaushalte eine Genossenschaft und verhandeln mit dem
Netzbetreiber. Statt eines niedrigeren Strombezugspreis (vgl. Yello
SmartMeter) kann der Preis hier freier verhandelt werden. Der Vertrieb
der Stromversorger wird umgangen.

Gleichzeitig sind hier natürlich Probleme des Systemdesigns zu beachten:
Als großes verteiltes System muss die Umsetzung zu einem stabilen
Systemverhalten führen. Ausfälle von einzelnen Systemkomponenten dürfen
nicht zu Störungen des Gesamtsystems führen.

Wir favorisieren dabei einen dezentralen Ansatz: Steuergeräte in den
einzelnen Haushalten verhalten sich dabei autonom. Lediglich ein
gewünschtes Lastprofil wird den Haushalten vorgegeben. Die einzelnen
Haushalte entscheiden dann autonom, wie einzelne Geräte zu steuern sind.
Die Verteilung des Lastprofils kann dabei durch standardisierte
Protokolle über das Internet erfolgen.

TODO: Modellierung von verschiedenen Lastgängen durch die Kombination
von Schaltzeitpunkten und Gruppen von Haushalten / Gewerbebetrieben.
Eventuell dynamische Anpassung der Gruppen, um gleiche ``Steuergrößen''
pro Gruppe abrufen zu können. 

