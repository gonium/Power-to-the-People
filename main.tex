%/* vim: set filetype=latex */

% Demo project. Uses Komascript >=3.0. 
% which is not included in texlive 2008
% (1) copy dist to texlive/texmf-local 
% (2) texhash
%\documentclass[BCOR=8.5mm,DIV=calc,open=right,pagesize=auto,a5paper]{scrbook}
%\documentclass[12pt,BCOR=8.5mm]{scrbook}
\documentclass[12pt,BCOR=8.5mm]{scrartcl}
\title{Power to the People}
\subtitle{Warum das Stromnetz Open-Source braucht}
\author{Mathias Dalheimer}
%% PDF SETUP
\usepackage[pdftex, bookmarks, colorlinks, breaklinks,
pdftitle=\title,pdfauthor=\author,plainpages=false]{hyperref}  
\hypersetup{linkcolor=blue,citecolor=blue,filecolor=black,urlcolor=blue,plainpages=false} 

\clubpenalty=3000 % adjust for widows and orphans 10000 is max
\widowpenalty=3000 % adjust for widows and orphans 10000 is max
%% Font Adjustments
%\usepackage[LY1]{fontenc}
\usepackage[T1]{fontenc}
%\usepackage{adobegaramond}
%\usepackage{gillsans}
%\renewcommand{\rmdefault}{HoeflerText}

% Use URW Garamond No. 8 as a default font. (getnonfreefonts)
\usepackage[urw-garamond]{mathdesign}
%\renewcommand{\rmdefault}{ugm}
% Optima as a sans serif font.
\renewcommand*\sfdefault{uop}

\newcommand*\imgwidth{0.8\textwidth}

\usepackage{ifthen}
\newboolean{InternalVersion}
\setboolean{InternalVersion}{true}
%\setboolean{InternalVersion}{false}

\usepackage{listings}  
\lstset{numbers=left, numberstyle=\tiny,numbersep=5pt}  
\lstset{language=Xml, basicstyle=\small, frame=shadowbox}  

\usepackage{algorithmic}
\usepackage{algorithm}
%\numberwithin{algorithm}{chapter} 
%\newcommand{\theHalgorithm}{\arabic{algorithm}}

\usepackage[protrusion=true,expansion=true]{microtype}

%% Page Header
\usepackage{scrpage2}

% Recalculate page setup based on new font.
\KOMAoptions{DIV=last}
%\KOMAoptions{draft=true}

% Versals. 
%\usepackage{lettrine}
% Misc packages
\usepackage{url}
\usepackage{graphicx}
\usepackage{todonotes}
%\usepackage[english]{babel}
\usepackage{ngerman}
% Use utf-8 encoding for foreign characters
\usepackage[utf8]{inputenc}
\usepackage{blindtext}
\usepackage{subfigure}

\hyphenation{scho-oner Sur-vivor ap-pen-dix}

%% Chapterstyle
%\renewcommand*{\chapterformat}{---\hskip.5cm\thechapter\hskip.5cm---}
%\KOMAoption{headings}{small,twolinechapter}
%\setkomafont{chapter}{\Large\sffamily\centering}
%\setkomafont{dictum}{\normalfont}
%\renewcommand*{\dictumwidth}{.8\textwidth}

%% Main.
\begin{document}
\maketitle
\clearscrheadfoot
\automark[chapter]{section}
\lehead[]{\pagemark\hskip.5cm\vrule\hskip.5cm\title}
\rohead[]{\headmark\hskip.5cm\vrule\hskip.5cm\pagemark} 
\title
%\pagestyle{scrplain}
%\begin{titlepage}
%%/* vim: set filetype=tex */

%\vspace*{\baselineskip} 
\vfill 
\hbox{% 
\hspace*{0.2\textwidth}% 
\rule{1pt}{\textheight} 
\hspace*{0.05\textwidth}% 
\parbox[b]{0.75\textwidth}{ 
\vbox{% 
%\vspace{0.1\textheight} 
{\noindent\huge\sffamily \title\\[0.5\baselineskip] }
\\[2\baselineskip] 
{\Large\sffamily \subtitle}\\[4\baselineskip] 
{\Large\sffamily \author }
\\[\baselineskip] 
{\Large\sffamily dalheimer@itwm.fhg.de}\par
\vspace{0.5\textheight} 
{\noindent Auf dem Weg zum Energiesystem für die nächsten 50 Jahre}\\[\baselineskip] 
}% end of vbox 
}% end of parbox 
}% end of hbox 
\vfill 
%\null 

%\end{titlepage}
%% frontmatter
%%\input{frontmatter}
%%\parindent.0cm
%%\parskip.4cm
%%\input{pagetwo}
\parindent.4cm
\parskip.0cm
\pagestyle{scrheadings}

\section{Einleitung}
Wenn morgens in Deutschland die Kaffeemaschinen angeschaltet werden,
sorgt ein komplexes System dafür, dass der Tag gut anfängt: Unser
Stromnetz. Das deutsche Stromnetz ist über die vergangenen 100 Jahre
gewachsen und transportiert den Strom von Kraftwerken zu den
Verbrauchern. Bildlich kann man sich das anhand des "`Stromsees"' vor
Augen führen:

\todo{Abbildung Stromsee}

Die einzelnen Kraftwerke speisen die elektrische Energie in das Netz
ein. Dort wird die Energie gesammelt, bis ein beliebiger Verbraucher den
Strom benötigt\footnote{Streng genommen wird natürlich Strom nicht
verbraucht, ich bleibe allerdings bei dieser umgangssprachlichen
Formulierung.}. Konventionelle Kohlekraftwerke tragen genauso wie
Atomkraftwerke und Photovoltaiksysteme zur Stromerzeugung bei. So
anschaulich dieses Modell ist, so vereinfacht es leider zu stark:

\begin{enumerate}
  \item Stromnetz kein speicher, \begin{equation}
      \textrm{Erzeugung}(t) = \textrm{Verbrauch}(t) \forall t
    \end{equation}
  \item Erzeugung ist geographisch verteilt, sprich:
    Höchstspannungsnetze etc. dazwischen. Es ist nicht möglich, die
    Leistung an beliebiger Stelle einzuspeisen.
    Ebenen des Stromnetzes erläutern, Niederspannungsnetz unten
    verwendet.
  \item Die Organisation des Strommarktes: Netzbetrieb vs. Vertrieb,
    getrennte Unternehmen.
\end{enumerate}

Die generative Stromerzeugung aus Wind, Photovoltaik und Biomasse hat
einen steigenden Anteil. Im Jahr 2009 betrug der Anteil 14,9\%, Tendenz
weiter steigend.\todo{Quelle} Grundsätzlich macht dieser Trend aus ökologischer
Sicht Sinn. Gleichzeitig verändert er jedoch die Annahmen, unter
denen unser Stromnetz gebaut wurde: Erzeugung zentralisieren und den
Strom zum Verbraucher transportieren. Vor allem 
Photovoltaikinstallationen sind geographisch stark verteilt, was zu
einer Einspeisung von Strom in die Niederspannungsebene führt. Diese
wurde nicht dafür ausgelegt, den Strom entgegenzunehmen und überregional
zu verteilen. Im Gegenteil, historisch wurde es als effizienter
betrachtet, die Stromerzeugung zu zentralisieren und auf kleinere,
geographisch verteilte Kraftwerke z.B. zur Versorgung von einzelnen
Industriebetrieben zu verzichten.\todo{Referenc Carr, Cloud Computing}

Ein weiteres Problem ist die unregelmässige Erzeugung: Während man ein
Gaskraftwerk nach Bedarf hoch- oder herunterregeln kann, ist dies bei
den generativen Stromerzeugern nicht der Fall. Wenn der Wind weht,
erzeugen die in Betrieb befindlichen Anlagen \todo{Genaue Zahl + Quelle}
GWh. \todo{Gesamtbedarf BRD gegenüberstellen}. Die Konsequenz: Um den
Strom in Deutschland aufnehmen zu können, müssten andere Anlagen wie
Kernkraftwerke abgeschaltet werden. Dies ist jedoch im Verlauf eines
Tages nicht realisierbar. Im Gegenteil, im Moment werden
Windkraftanlagen angehalten, weil andere Kraftwerke nicht schnell genug
heruntergeregelt werden können\footnote{Dies ist auch der Grund, warum die
Kernenergie keinesfalls als Brückentechnologie die Erneuerbaren
unterstützt.}. 

Zusammenfassend lässt sich festhalten: Erneuerbare Energien --- so
wünschenswert ihr Einsatz ist --- führen zu Problemen im Stromnetz. Die
etablierten Player am Markt haben darüber hinaus auch kein Interesse an
einer verteilten Einspeisung von Strom, da diese ihre marktbeherrschende
Position schwächen würde. Wohin also mit dem "`grünen"' Strom?

\section{Flexibel durch Demand-Side Management}\label{sec:demand-side_management}

Unser Stromnetz muss flexibler werden, um diese Stromeinspeisung
aufnehmen zu können. Auf der Seite des Netzbetriebs und des Ausbaus der
Stromnetze können wir als Open-Source Community nur einen indirekten
Beitrag liefern, insofern gehe ich hier nicht näher darauf ein. Auch für
die Speicherung von Strom in Pumpspeicherkraftwerken und --- neuerdings
--- Batterien sind erhebliche Investitionen nötig, sodass sich auch hier
keine Spielwiese findet.

Eine andere Herangehensweise ist jedoch das \emph{Demand-Side
Management}. Statt auf der Seite der Stromerzeugung und -verteilung
Änderungen vorzunehmen, wird der Verbrauch von Strom beeinflusst. Damit
ist jedoch zunächst nicht die Reduzierung des Stromverbrauchs gemeint,
obwohl dies natürlich eine sinnvolle Anstrengung ist. Stattdessen wird
der Stromverbrauch auf der Zeitachse verschoben. Das Ziel ist es,
Stromverbraucher dann zu betreiben, wenn sowieso viel Strom erzeugt
wird. Umgekehrt laufen diese Verbraucher nicht, wenn zu einem anderen
Zeitpunkt weniger Strom erzeugt wird.
\todo{Bilder Strom Shaping, vgl. E-Energy Studie}
\todo{Impact, Siehe BEE-Studie}

Im Industriebereich ist diese Herangehensweise schon lange Standard ---
unter dem Stichwort "`Lastabwurf"' hat ein Netzbetreiber die
Möglichkeit, den Stromverbrauch von einzelnen Industriebetrieben gezielt
zu reduzieren, um Engpässe zu vermeiden. Im Gegenzug erhält der
Industriebetrieb bessere Strompreise. Ein Netzbetreiber ist so in der
Lage, relativ schnell erhebliche Lasten im Netz der momentanen Erzeugung
anzupassen.

Im Privathaushalt ist diese Technik jedoch noch nicht im Einsatz. Die
Gründe hierfür sind vielfältig: Einerseits ist die notwendige Technik
noch nicht weit verbreitet (Stichwort: Hausbussysteme), andererseits hat
es sich bisher auch finanziell nicht gelohnt. Allerdings ist absehbar,
dass nun die Zeit reif für Demand Side Management in Haushalten ist. Das
zeigt sich auch daran, dass im Rahmen der E-Energy Initiative der
Bundesregierung viele Projekte an dieser Idee arbeiten\todo{Referenz
einfpgen}. Leider sind viele diese Projekte jedoch von den großen
Stromkonzernen dominiert, sodass bezweifelt werden darf, dass die
Projekte zu kundenorientierten Lösungen führen werden. 

Dies wird zum Beispiel an der Datenschutzproblematik im Umgang mit Smart
Metern deutlich. Ein Smart Meter ist ein Stromzähler, der die Messwerte
in kurzen Zeitintervallen (typisch sind 15 Minuten) an den Netzbetreiber
überträgt und auch dem Stromkunden zugänglich macht. Einerseits sind
diese Daten natürlich für den Stromkunden interessant, da er den
Einfluss seines Verhaltens auf den Stromverbrauch vor Augen geführt
bekommt und so insgesamt weniger verbrauchen wird.

Andererseits sind die Stromverbrauchsdaten aus Netzbetreibersicht auch
sehr interessant, da sich hier völlig neue Möglichkeiten für
Preismodelle und auch für das Marketing ergeben. Dazu muss man wissen,
dass Privathaushalte im Moment durch Standardlastprofile abgerechnet
werden, d.h. ein Netzbetreiber wird nicht für die real gelieferte
Strommenge bezahlt, sondern auf der Basis eines durchschnittlichen
Lastprofils und der Anzahl der versorgten Haushalte wird eine Pauschale
abgerechnet. Mit Smart Metern kann nun der reale Verbrauch bestimmt
werden und wird in Zukunft wohl zur Grundlage der Abrechnung werden
\footnote{Entsprechende Änderungen werden derzeit von der
Bundesnetzagentur diskutiert.}. 

Um diese Abrechnung vorzunehmen gehen die Netzbetreiber davon aus, dass
die Stromverbrauchsdaten in hoher zeitlicher Auflösung an sie
übertragen werden, um dann eine verbrauchsgenaue Abrechnung gegenüber
den Stromanbietern machen zu können\todo{Struktur Strommarkt erläutern}.
Diese Daten sind jedoch als sehr sensibel einzustufen, da hieraus auf
Lebens- und Konsumgewohnheiten geschlossen werden kann.
\todo{Bild Tagesrhythmus, Artikel KJM, FAZ-Artikel}

Die Netzbetreiber versuchen sich im Moment dadurch abzusichern, dass die
Stromkunden ihnen eine Einwilligung geben. Dies ist jedoch juristisch
eine sehr optimistische Herangehensweise \todo{Artikel DuD, Rechtliche
Bewertung} und darüber hinaus eine grundlegend abzulehnende Forderung.
Jeder Zähler verfügt über mehrere Register, in denen der Stromverbrauch
zu unterschiedlichen Tarifzeiten abgelegt werden kann --- eine zeitnahe
Übertragung in hoher Auflösung ist schlicht nicht erforderlich.

\section{Chancen für Open-Source}\label{sec:chancen_open-source}
In kundenfreundlichen Lösungen liegt die Chance für Open-Source
Technologien. Indem die Technologien direkt von den Anwendern entwickelt
werden, sind die Chancen groß, dass der Nutzen für Privathaushalte im
Vordergrund steht.  Ich denke auch, dass es in Zusammenarbeit mit
kleinen Stadtwerken in öffentlicher Hand möglich ist, eine
Strommanagement-Infrastruktur kostendeckend einzuführen.

Im Projekt \emph{mySmartGrid} wird zur Zeit in Kaiserslautern eine
entsprechende Infrastruktur aufgebaut. Das Projekt wird vom Land
Rheinland-Pfalz im Rahmen des Konjunkturprogramm II gefördert. Alle
Technologien werden in enger Zusammenarbeit sowohl mit den
Stromverbrauchern als auch mit den Netzbetreibern aus der Region
(Technische Werke Kaiserslautern, Pfalzwerke) entwickelt.  Wir bauen auf
ein Ökosystem von freien Lösungen.  Bei der Hardware setzen wir auf
Consumer-Geräte, die einen hohen Verbreitungsgrad haben.  Ein möglichst
großer Anteil der Funktionen soll in Software implementiert werden, da
diese quasi ohne weitere Kosten vervielfältigt werden kann.
Selbstverständlich werden alle Projektresultate (Soft- und Hardware)
unter einer Open-Source Lizenz veröffentlicht. 

Gleichzeitig werden ca.  1000 Haushalte in Kaiserslautern und Umgebung
mit der Technik ausgerüstet. Da die Installation von Geräten durch
Handwerker aus der Region erfolgt, ist eine möglichst gute
Installationsunterstützung durch Softwarewerkzeuge notwendig.

Das Projekt ist in drei Phasen gegliedert, die ich im Folgenden
darstelle.

\subsection{Messen und Verstehen}\label{sub:messenverstehen}

In der ersten Projektphase geht es darum, den Teilnehmern ein
Verständnis für den individuellen Stromverbrauch zu geben. Hierzu
benutzen wir den "`Flukso"', um den den Stromverbrauch eines Haushalts
zentral zu messen.\todo{Bild Flukso einfügen, Link}. Technisch ist der
Flukso ein WLAN-Router, basiert auf OpenWRT und benutzt einen kleinen
Mikrocontroller zur Messwerterfassung. Die Installation erfordert keine
Neuverkabelung, da die Messung indirekt über Halleffektsensoren erfolgt.
Die Messwerte werden zur mySmartGrid-Webseite übertragen, wo sie
visualisiert werden können. Ausserdem können die Messwerte auch lokal
abgefragt werden.

Die Daten werden auf der mySmartGrid-Webseite nicht dauerhaft
gespeichert, sondern nach und nach vergessen. Für die erste Stunde sind
Minutenwerte gespeichert, für die letzten 24 Stunden nur noch
fünfminütige Werte. Danach sinkt die Auflösung rapide. Wer die Daten
dennoch in hoher Auflösung haben möchte, muss die API der Webseite
benutzen. Weil viele Teilnehmer dies tun möchten, haben wir auch einen
"`Rekorder-Dienst"' implementiert, welcher die Daten in höchster
Auflösung mitschreiben kann. 

Es sind auch Open-Source Alternativen zum Flukso vorhanden, zum Beispiel
der Volkszähler \todo{link}. Obwohl wir ihn in diesem Projekt nicht einsetzen, sind
im Moment Diskussionen im Gange, die Schnittstellen des Fluksos und des
Volkszählers kompatibel zu machen. Aber auch die kommerziellen und von
den Netzbetreibern eingesetzten Smart Meter müssen eine lokal
auslesbare, optische Schnittstelle bieten \todo{Quelle: Positionspapier
Bundesnetzagentur}.  Die Umsetzung dieser Schnittstelle bleibt jedoch
zunächst dem Hersteller überlassen. Viele Hersteller setzen auf den
elektronischen Einheitszähler, der vom VDE \todo{Referenz
Netzkompetenzzentrum} standardisiert wird. Über die optische serielle
Schnittstelle wird jede Sekunde ein Datagramm mit allen relevanten
Informationen ausgesandt. Ein Mikrocontroller mit zugehöriger Fotodiode
ist prinzipiell alles, was zum Auslesen benötigt wird --- eine
Open-Source Lösung hierfür fehlt jedoch derzeit.

Das Messen ist allerdings zumeist nicht genug, um eine nachhaltige
Änderung des Stromverbrauchsverhaltens anzuregen. Dazu setzen wir auf
zwei Ansätze:

\begin{enumerate}
  \item Einerseits helfen automatisierte Analysen, den eigenen
    Stromverbrauch zu beurteilen. Dazu sind zunächst recht fein
    aufgelöste Stromverbrauchswerte nötig, um einzelne Verbraucher
    erkennen zu können. Daher muss dieses Verfahren optional bleiben und
    nur bei Bedarf vorgenommen werden. Wir schneiden dann für zwei
    Wochen die Stromverbrauchswerte mit und identifizieren einzelne
    Verbraucher. Basierend auf dieser Analyse ist es dann recht einfach,
    konkrete Handlungsempfehlungen zu geben, z.B. "`Tauschen Sie Ihren
    Kühlschrank aus --- das amortisiert sich nach $2,4$ Jahren."' 
    Diese Analysen sind im Moment noch im Forschungsstadium.
  \item Andererseits ist es für Stromkunden hilfreich, den Effekt ihrer
    Handlungen unmittelbar zu sehen. Wenn ich den Wasserkocher
    einschalte, dann verbrauche ich auf einmal 2kW - das führt zu einem
    gut sichtbaren Sprung in der Stromverbrauchskurve. Die logische
    Konsequenz wäre es, nur die wirklich benötigte Menge Wasser in den
    Wasserkocher zu füllen.
\end{enumerate}

Zur Anzeige der Verbrauchsinformationen haben wir uns für den
"`Chumby"` entschieden\todo{Bild}. Der Chumby basiert technisch
ebenfalls auf einem WLAN-Router, hat allerdings einen Touchscreen und
einen Lautsprecher integriert. Es sind schon viele Applikationen für den
Chumby verfügbar, sodass das Gerät neben Stromverbrauchsinformationen
auch den Wetterbericht anzeigen und Internetradio abspielen kann. Unsere
Nutzer können sich die Applikationen selbst zusammenstellen. Welche
Darstellungsformen für die Strominformationen am Sinnvollsten sind,
werden wir in der kommenden Zeit zusammen mit unseren Projektteilnehmern
erarbeiten.

\subsection{Regelung von Verbrauchern}\label{sub:steuern}

Der nächste Schritt ist das Steuern von Verbrauchern im Haushalt.
Mit Hilfe von meteorologischen Modellen ist es möglich, die
Stromproduktion auf Tagesfrist recht genau vorherzusagen. Wenn also
morgen Mittag der Wind weht, führt dies auch zu einer erhöhten
Stromproduktion. Diese Information kann dann einen Tag im Voraus zu den
Haushalten transportiert werden. Die Nutzer können dann diese
Information angezeigt bekommen und ihr Verhalten entsprechend anpassen
--- für eine Spülmaschine ist es meist recht egal, ob sie morgens oder
mittags läuft.

Natürlich ist dies aber nicht genug, denn eigentlich sollte diese
Regelung automatisch passieren. Leider bieten die meisten
Haushaltsgeräte keine Möglichkeit, Steuerbefehle à la "`heute mittag um
14:00 Strom verbrauchen"' zu verarbeiten. \footnote{Meines Wissens nach
gibt es nur eine Waschmaschine von Miele, die es überhaupt erlaubt, in
ein Hausbussystem integriert zu werden.} Ebenso ungelöst ist die Frage
nach einem universellen und nachrüstbaren Bussystem, welches diese
Steuerinformationen zu den Verbrauchern transportiert. Bussysteme wie
EIB und KNX sind nur dann sinnvoll einsetzbar, wenn gleichzeitig größere
Umbauten an der Wohnung vorgenommen werden. Darüber hinaus sind diese
Systeme auch recht teuer. Dies schließt alle Leute aus, die in einer
Mietwohnung wohnen, denn beim Auszug ist es nicht ohne weiteres
möglich, das Bussystem mitzunehmen.

Mit "`digitalStrom"' wird derzeit ein anderes System entwickelt, welches
mit relativ geringem Aufwand nachrüstbar ist und Steuerinformationen
über das Stromnetz transportiert. Dieses System ist jedoch noch nicht am
Markt verfügbar.

Langfristig ist eine Lösung wünschenswert, die eine IP-basierte
Kommunikationsinfrastruktur auf kleinen Mikrocontrollern umsetzt, damit
die Lösung bezahlbar bleibt. Meine eigenen Experimente mit 6LoWPAN und
funkbasierter Kommunikation auf der Basis von 868 MHz zeigen, dass eine
solche Lösung sehr wohl kostengünstig umzusetzen ist\todo{Referenz
Octobus, Bild}. Mit Conkiki \todo{Referenz} steht auch ein Open-Source
Betriebssystem zur Verfügung. Alternativ ist auch mit Ethersex
\todo{Referenz} eine etablierte Lösung vorhanden. 

Um die Entwicklung voran zu treiben setzen wir im Projekt mySmartGrid
auf die Steuerung von Wärmespeichern wie Kühlgeräte und Wärmepumpen. Wir
greifen dabei nicht in die interne Regelung der Geräte
ein\footnote{Es wäre für unsere Teilnehmer wohl nicht akzeptabel, wenn
ich mit meinem Lötkolben an ihren Kühlschränken herumbastele ;-)}. Für
Kühlschränke und Gefriertruhen bevorzugen wir folgenden Ansatz: Die
Geräte werden über einen Zwischenstecker von der Stromversorgung
getrennt. In der Folge steigt die Innentemperatur an. Wenn nun der
Stromverbrauch zu einem Zeitpunkt $t$ maximiert werden soll, muss die
Innentemperatur zu diesem Zeitpunkt also recht hoch sein. Dann wird das
Kühlgerät wieder ans Netz angeschlossen. Die Regelung des Kühlgerätes
wird nun aufgrund der (relativ) hohen Innentemperatur den Kompressor
anschalten und wie gewünscht Strom verbrauchen. Diese Herangehensweise
darf jedoch nicht dazu führen, dass Lebensmittel verderben oder
Tiefkühlware auftaut.

Eine einfache Lösung wäre, die Innenraumtemperatur des Kühlgerätes über
einen Sensor zu überwachen. Dieser Sensor verursacht jedoch
Zusatzkosten. Daher prognostizieren wir das Verhalten der internen
Regelung des Gerätes und übersteuern diese gezielt. Das setzt eine
Systemidentifikation und eine Einrichtungsprozedur voraus. Diese
ermittelt dann Regelparameter, die dazu benutzt werden, das Kühlgerät
gezielt vor dem geplanten Anschaltzeitpunkt auszuschalten. Gleichzeitig
ist gewährleistet, dass die Kühlkette nicht unterbrochen wird. Aus den
Strommessungen lässt sich lokal auch feststellen, ob der Kompressor
eines Kühlgerätes anläuft. Durch kurzes Einschalten ist es auch ohne
Temperatursensor möglich, Rückschlüsse auf die Innentemperatur des
Kühlgerätes zu ziehen.


Der wichtigste Faktor ist jedoch die Akzeptanz der Technologie bei den
Anwendern: Diese müssen jederzeit in der Lage sein, die vorgeschlagenen
Regeleingriffe abzulehnen. Da die Regelungsalgorithmen sowieso auf dem
Chumby laufen werden, ist hier auch der logische Platz, um den Benutzer
über die aktuelle Planung zu informieren. Dort wird es dann auch die
Möglichkeit geben, die Regelung zu beeinflussen oder auch zu
deaktivieren.

Schliesslich ist eine weitere Gruppe von Stromkunden sehr interessant
für den Einsatz von Haussteuerungen: Für Photovoltaikanlagenbesitzer ist ab
diesem Sommer der Eigenverbrauch des selbst erzeugten Stroms die
rentabelste Option\todo{Kleines Rechenbeispiel}. Dies macht auch
ökonomisch Sinn, denn der selbstverbrauchte Strom muss nicht über das
Stromnetz transportiert werden und entlastet es. Ein teurer Netzausbau
kann so vielleicht nicht ganz vermieden, aber doch verzögert bzw.
reduziert werden.

Hier ist es wiederum möglich, die Stromproduktion für den folgenden Tag
recht genau vorherzusagen. Neben der Anzeige auf dem Chumby können auch
Geräte so gesteuert werden, dass sie den selbst produzierten Solarstrom
direkt aufnehmen.

\subsection{Virtueller Verbraucher}\label{sub:virtuellerverbraucher}
Natürlich ist es für einen einzelnen Haushalt nicht möglich,
signifikanten Einfluss auf das Stromnetz zu haben. Allerdings gibt es ja
recht viele Haushalte, auf die man potentiell Einfluss nehmen könnte.
Alle Teilnehmer rufen also einen Tag im Voraus die Prognose für den
nächsten Tag ab und versuchen, diese umzusetzen. Die genaue Modellierung
ist hier ein statistisches Problem, denn nicht jeder Haushalt wird sich
an die Handlungsempfehlungen halten. Zusammengenommen dürfte es jedoch
möglich sein, einen signifikanten Einfluss auf das lokale Stromnetz zu
haben.

Dieser Eingriff bietet Chancen für die lokalen Netzbetreiber. Diese
können dieses Regelpotential in ihre kurzfristige Planung mit
einbeziehen. Wenn normalerweise Lastspitzen durch den teuren,
kurzfristigen Zukauf von Strom abgedeckt werden müssen, können sie durch
die Verschiebung von Lasten Geld sparen. Diesen Profit können sie sich
dann mit den teilnehmenden Haushalten teilen. Hier sind zwei Modelle denkbar: 
\begin{enumerate}
  \item Die Haushalte schliessen sich zu einer Genossenschaft zusammen
    und verhandeln direkt mit dem lokalen Netzbetreiber. Die Vermarktung
    des Regelpotentials ist nicht an einen Stromlieferanten gebunden,
    d.h. die Eigner der Genossenschaft können ihren Strom bei
    unterschiedlichen Lieferanten einkaufen. Die Einnahmen der
    Genossenschaft können zur Finanzierung der Geräte verwendet werden.
    Dieses Modell macht dort Sinn, wo kleinere Netzbetreiber wie
    unabhängige Stadtwerke den Netzbetrieb organisieren.
  \item Ein anderes Modell wäre die Finanzierung der Geräte etc. über
    den Stromvertrieb, d.h. ein Stromkunde bekommt andere
    Lieferkonditionen, wenn er mit der Regelung seiner Geräte
    einverstanden ist. Hier hat der Stromkunde gegenüber dem
    Genossenschaftsmodell eine schwächere Position. Zudem ist dieses
    Modell aufgrund der Organisation des Strommarkts schwierig
    umzusetzen.
\end{enumerate}

Für mySmartGrid favorisieren wir das Genossenschaftsmodell. Unser Plan
ist es, die Geräte aus dem Projekt bei den Teilnehmern zu belassen und
die Gründung einer Genossenschaft anzustoßen. Die Herausforderung
bleibt: Wie kann Demand-Side Management kostendeckend eingesetzt werden?

\section{Schlussfolgerungen}
Smart Grids, "`intelligente Stromnetze"', sind eines der Themen, welche
von der Politik und natürlich auch der Stromwirtschaft immer wieder in
den Vordergrund gestellt werden. Leider kommt dabei der Stromkunde zu
kurz --- die Bedürfnisse von Stromkunden werden weitgehend ignoriert und
der Datenschutz wird oft ausser acht gelassen\todo{Referenz
BMWi-Konferenz Nutzerschutz}.

Aber auch kleinere Stadtwerke haben mit dieser Entwicklung Probleme:
Aufgrund politischer Vorgaben müssen sie zum Beispiel Smart Meter
einführen, obwohl ihnen dadurch Kosten entstehen, die sie nicht direkt
auf den Kunden umlegen können. 

Gleichzeitig ist die Einführung von "`intelligenten"' Technologien in
das Stromnetz eine hervorragende Spielwiese für alle Hacker und Nerds.
Hier gilt es, Gedanken umzusetzen, die sowieso in den Hackerspaces
dieser Welt diskutiert werden\footnote{Warum definiert jeder Hackerspace
ein eigenes Hausbussystem?}. Das Problem rein technologisch anzugehen
wäre allerdings zuwenig. Sowohl ökonomische, ökologische als auch
gesellschaftliche Überlegungen müssen mit einbezogen werden. Es wird
auch immer mehr als eine Lösung geben. Insofern sind die idealen
Voraussetzungen für ein Open-Source Ökosystem gegeben --- lasst uns diese
Spielwiese nutzen!

\cite{geller2010smartgrid}


%%/* vim: set filetype=tex */

\section{Einleitung}\label{sec:einleitung}

Regenerative Energien, Zuwachs in Zukunft
Studie BEE als Argument benutzen

Aber: Problem Produktion und Verbrauch von elektrischer Energie.
Grundannahme: Entnahme = Einspeisung zu jedem Zeitpunkt

\begin{enumerate}
  \item Grundlastkraftwerke: Vor allem Braun- und Steinkohlekraftwerke,
	aber auch Atomkraftwerke.
  \item Volatile Kraftwerke: Wind und Sonne, aber auch Gaskraftwerke.
	Unterscheidung in planbare und nicht planbare Erzeugung von
	Elektrizität.
\end{enumerate}

Erneuerbare-Energien-Gesetz: Der Netzbetreiber muss die Einspeisungen
von Solaranlagen entgegennehmen und vergüten. Das hat Folgen für
die Netzstabilität.

Business von Netzbetreibern: Stabilität des Netzes sicherstellen. Dabei
vom Vertrieb getrennt. Stromeinkauf einerseits langfristig (OTC),
andererseits kurzfristiger Zukauf von Ausgleichsenergie.

Vgl. auch Einleitung DuD (KJM)

TODO: Mit Bichler reden, um die Fakten richtig hinzubekommen.

Zentrale Frage: Wie kann man als Netzbetreiber mit volatiler Stromeinspeisung umgehen?
Traditioneller Weg: Zur Erzeugung von Ausgleichsstrom werden auch
Gaskraftwerke hoch- bzw. heruntergeregelt. Diese Kapazitäten werden in
Zukunf allerdings nicht ausreichen.

Im Prinzip zwei Möglichkeiten:
\begin{enumerate}
  \item Speicherung von Strom, z.B. in Pumpspeicherkraftwerken oder
	Batterien. Problem: Die Kapazitäten reichen nicht aus. 
  \item Demand-Side Management: Verlagerung von Lasten auf der
	Zeitachse.
\end{enumerate}

Schließlich führt die Erhebung von Stromverbrauchsinformationen auch zur
Veränderung von Konsumgewohnheiten. Stromkunden können, wenn
Informationen zum Momentanverbrauch unmittelbar zur Verfügung stehen,
ihr Verhalten direkt verändern und so ihren Strombezug um 15-20\%
reduzieren~\cite{geller2010smartgrid}. Ebenso ist es denkbar, mit den
gesammelten Strombezugsinformationen weiterführende Analysen
durchzuführen. Diese können zum Beispiel Verbraucher identifizieren, die
einen erheblichen Anteil am Stromverbrauch haben. Daraufhin können
automatisiert Hinweise gegeben werden, dass sich z.B. die Anschaffung
eines energiesparenden Kühlschranks schon nach einem Jahr amortisiert
hätte.


TODO: Grundlegende Terminologie festlegen.
Netzbetreiber, Verbraucher (gerät), Stromkunde, Strombezug (anstatt
Stromverbrauch)

%%/* vim: setformat tex
%*/

\section{Demand Side Management}\label{sec:demandside}
Natürlich ist es erstrebenswert, absolut gesehen weniger Strom zu
verbrauchen\footnote{Die Autoren sind sich darüber im klaren, dass Energie
nicht verbraucht werden kann --- wir folgen jedoch dem allgemeinen
Sprachgebrauch.}. Allerdings ist dies nicht das Ziel des Demand-Side
Managements. Stattdessen wird hier versucht, den Verbrauch an die
Erzeugung von Strom anzupassen. Dieses Verfahren ist seit langem in der
Industrie üblich --- großen Verbrauchern aus der Industrie werden
vergünstigte Preise eingeräumt, wenn dafür im Falle von Netzengpässen
der Stromverbraucher abgeschaltet werden können. 

Dazu teilt der Netzbetreiber dem Großverbraucher auf kurze Sicht (15
Minuten) mit, dass einige Aggregate abzuschalten sind. Daraufhin werden
Spitzen im Strombedarf gekappt.

TODO: Verschiedene Szenarien, siehe eEnergy-Studie

Das gleiche Grundprinzip kann natürlich auch im Haushalt umgesetzt
werden. Hier muss es jedoch das Ziel sein, eine kostengünstige Umsetzung
ohne Komforteinbußen für den Verbraucher zu realisieren. Dazu spielt
natürlich auch der Preis eine Rolle. Zudem ist der verschiebbare
Stromverbrauch eines einzelnen Verbrauchers nicht groß im Vergleich zu
der benötigten Ausgleichsenergie. Allerdings kann durch die Kombination
einer hinreichend großen Anzahl an Haushalten ein signifikanter
Verbrauch verschoben werden. Diese Ansammlung von Haushalten und
natürlich auch Gewerbebetrieben kann als eine Einheit angesprochen
werden, wenn sie zu einem \emph{virtuellen Verbraucher}
zusammengeschlossen werden.

TODO: Kommerzielles Modell. Wie kann Demand-Side Management im
Privathaushalt nachhaltig werden? Angedacht ist der Verkauf der
Ausgleichsenergie an den Netzbetreiber. Dieser müsste nicht zu hohen
Preisen Strom kaufen bzw. verkaufen. Diese Marge kann sich der
Netzbetreiber mit den Privathaushalten teilen. Dzu gründen die
Privathaushalte eine Genossenschaft und verhandeln mit dem
Netzbetreiber. Statt eines niedrigeren Strombezugspreis (vgl. Yello
SmartMeter) kann der Preis hier freier verhandelt werden. Der Vertrieb
der Stromversorger wird umgangen.

Gleichzeitig sind hier natürlich Probleme des Systemdesigns zu beachten:
Als großes verteiltes System muss die Umsetzung zu einem stabilen
Systemverhalten führen. Ausfälle von einzelnen Systemkomponenten dürfen
nicht zu Störungen des Gesamtsystems führen.

Wir favorisieren dabei einen dezentralen Ansatz: Steuergeräte in den
einzelnen Haushalten verhalten sich dabei autonom. Lediglich ein
gewünschtes Lastprofil wird den Haushalten vorgegeben. Die einzelnen
Haushalte entscheiden dann autonom, wie einzelne Geräte zu steuern sind.
Die Verteilung des Lastprofils kann dabei durch standardisierte
Protokolle über das Internet erfolgen.

TODO: Modellierung von verschiedenen Lastgängen durch die Kombination
von Schaltzeitpunkten und Gruppen von Haushalten / Gewerbebetrieben.
Eventuell dynamische Anpassung der Gruppen, um gleiche ``Steuergrößen''
pro Gruppe abrufen zu können. 


%%/* vim: setformat tex
%*/

\section{Umsetzung}\label{sec:umsetzung}

Hier besteht die Gefahr, das Feld komplett den kommerziellen Anbietern
zu überlassen. Dies ist nicht unbedingt im Sinne der privaten
Verbraucher. Gleichzeitig werden auch kleinere Stadtwerke im Wettbewerb
übervorteilt, wenn große Stromanbieter ihre Infrastrukturen auf ihren
eigenen Gewinn hin optimieren. 

Wir gehen daher einen anderen Weg und 

%%/* vim: set filetype=tex */

\section{Risiken}\label{sec:risiken}
Die Einführung von SmartMetern birgt erhebliche Risiken für die
Privatsphäre. Definition Privacy vs. Security

Quellen: Google Powermeter, Hersteller, Erfassung von
15-Minuten-Interfaces
Stellungnahme der Bundesnetzagentur zu den §21 EnGG. Welche Kriterien
sind tatsächlich nötig?
DuD-Artikel

So weisen Narayanan und Shmtikov auf die Möglichkeiten der
``De-Anonymisierung'' von Datensätzen hin~\cite{narayanan2010pii}. Sie
beschreiben dabei die prinzipielle Unmöglichkeit, einmal erhobene Daten
zu anonymisieren und dabei die Rückverfolgbarkeit auszuschliessen.
Ebenso argumentiert Shapiro und weist darauf hin, dass Privacy als
nichtfunktionale Anforderung nur als Designkriterium am Anfang in die
Systementwicklung integriert werden kann~\cite{shapiro2010privacy}. Eine
nachträgliche Integration von Privacy in ein bereits existierendes
System ist quasi nicht möglich.

Fazit: Einmal in Umlauf gebrachte Daten können wieder auf individuelle
Verbraucher zurückgeführt werden.


\begin{enumerate}
  \item Zeige \& erkläre einen Tagesverlauf, zeige, dass daraus sehr
	persönliche Informationen extrahiert werden können.
  \item Diskutiere rechtliche Rahmenbedingungen: DuD-Artikel, Gutachten
	ULD Schleswig-Holstein -> gesetzliche Rahmenbedingungen müssen
	geschaffen werden.
\end{enumerate}

Durch die Kontrolltechnik werden DDOS-Attacken auf das Stromnetz möglich
-> Alle Verbraucher könnten zum gleichen Zeitpunkt angeschaltet werden.
Dies ist eine Herausforderung für Systemdesigner.

Für die Netzbetreiber / Messtellenbetreiber ist dies eine
Herausforderung, da Kunden nicht bereit sind, für die neue Messtechnik
zu bezahlen. Daher müssen Mehrwerte geschaffen werden.

Kunden müssen diesen Mehrwert sehen und bereit sein, in diese Technik zu
investieren. Dies kann natürlich auch komplett an den Stromversorgern
vorbei passieren, siehe oben: Wenn die mySmartGrid-Technologie
installiert ist, kann eine Genossenschaft gegründet werden.

%\section{Weitere Anwendungen}\label{sec:weitere_anwendungen}
Die Investitionen in das intelligente Stromnetz können auch eine Reihe
von weiteren Applikationen fördern, bzw. diese Applikationen
beschleunigen die Umsetzung von Smart Grid Technologien:

\begin{enumerate}
  \item Home Automation: Wenn ein Hausbussystem installiert wurde,
	können einerseits natürlich Steuerinformationen aus dem Stromnetz
	transportiert werden. Andererseits können diese Bussysteme natürlich
	auch für Komfortfunktionen etc. verwendet werden. Hier ergeben sich
	im besten Falle auch Synergien. Problematisch ist jedoch die
	fehlende Standardisierung und Nachrüstbarkeit in Mietwohnungen.
	Ebenfalls stellt sich die Frage, wie Hausbussysteme im Bestand
	kostengünstig integriert werden können.
  \item Assisted Living: Hier können Smart Meter eine interessante
	Ergänzung zu vielen integrierten Sensoren liefern. Die Zeiten
	zwischen den Aktivitäten von elektrischen Geräten geben Auskunft
	über die Aktivitäten der Bewohner.
  \item Eigenverbrauch von Photovoltaik-Strom optimieren
  \item Analyse und Überwachung von Photovoltaik-Anlagen
	(herstellerunabhängig)
\end{enumerate}




%\section{Ökosystem}\label{sec:ökosystem}
Ziel einerseits, Entwicker in die Infrastrukturentwicklung einzubinden,
andererseits auch, Verbrauchergemeinschaften zu bilden. Da alle
Komponenten als Open-Source Software und Hardware von jedem modifiziert
werden können, erwarten wir, dass viele induviduelle Anpassungen
entstehen werden.


\begin{enumerate}
  \item Viel Raum für eigene Ideen (Spieltrieb)
  \item Grundlage: Billige Vernetzung, Preise müssen weiter sinken
  \item NAchrüstbarkeit der Technologie in existierenden Wohnungen
  \item Regelungsalgorithmen bzw. Adapter für Geräte
  \item Analyse von Stromverbrauchsmustern
\end{enumerate}


\bibliographystyle{plain}
\bibliography{main}
\end{document}

% usage of todo package:
%\todo{foobar.}
%\missingfigure{A big black toad.}


