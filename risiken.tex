%/* vim: set filetype=tex */

\section{Risiken}\label{sec:risiken}
Die Einführung von SmartMetern birgt erhebliche Risiken für die
Privatsphäre. Definition Privacy vs. Security

Quellen: Google Powermeter, Hersteller, Erfassung von
15-Minuten-Interfaces
Stellungnahme der Bundesnetzagentur zu den §21 EnGG. Welche Kriterien
sind tatsächlich nötig?
DuD-Artikel

So weisen Narayanan und Shmtikov auf die Möglichkeiten der
``De-Anonymisierung'' von Datensätzen hin~\cite{narayanan2010pii}. Sie
beschreiben dabei die prinzipielle Unmöglichkeit, einmal erhobene Daten
zu anonymisieren und dabei die Rückverfolgbarkeit auszuschliessen.
Ebenso argumentiert Shapiro und weist darauf hin, dass Privacy als
nichtfunktionale Anforderung nur als Designkriterium am Anfang in die
Systementwicklung integriert werden kann~\cite{shapiro2010privacy}. Eine
nachträgliche Integration von Privacy in ein bereits existierendes
System ist quasi nicht möglich.

Fazit: Einmal in Umlauf gebrachte Daten können wieder auf individuelle
Verbraucher zurückgeführt werden.


\begin{enumerate}
  \item Zeige \& erkläre einen Tagesverlauf, zeige, dass daraus sehr
	persönliche Informationen extrahiert werden können.
  \item Diskutiere rechtliche Rahmenbedingungen: DuD-Artikel, Gutachten
	ULD Schleswig-Holstein -> gesetzliche Rahmenbedingungen müssen
	geschaffen werden.
\end{enumerate}

Durch die Kontrolltechnik werden DDOS-Attacken auf das Stromnetz möglich
-> Alle Verbraucher könnten zum gleichen Zeitpunkt angeschaltet werden.
Dies ist eine Herausforderung für Systemdesigner.

Für die Netzbetreiber / Messtellenbetreiber ist dies eine
Herausforderung, da Kunden nicht bereit sind, für die neue Messtechnik
zu bezahlen. Daher müssen Mehrwerte geschaffen werden.

Kunden müssen diesen Mehrwert sehen und bereit sein, in diese Technik zu
investieren. Dies kann natürlich auch komplett an den Stromversorgern
vorbei passieren, siehe oben: Wenn die mySmartGrid-Technologie
installiert ist, kann eine Genossenschaft gegründet werden.
