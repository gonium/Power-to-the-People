%/* vim: set filetype=tex */

\section{Einleitung}\label{sec:einleitung}

Regenerative Energien, Zuwachs in Zukunft
Studie BEE als Argument benutzen

Aber: Problem Produktion und Verbrauch von elektrischer Energie.
Grundannahme: Entnahme = Einspeisung zu jedem Zeitpunkt

\begin{enumerate}
  \item Grundlastkraftwerke: Vor allem Braun- und Steinkohlekraftwerke,
	aber auch Atomkraftwerke.
  \item Volatile Kraftwerke: Wind und Sonne, aber auch Gaskraftwerke.
	Unterscheidung in planbare und nicht planbare Erzeugung von
	Elektrizität.
\end{enumerate}

Erneuerbare-Energien-Gesetz: Der Netzbetreiber muss die Einspeisungen
von Solaranlagen entgegennehmen und vergüten. Das hat Folgen für
die Netzstabilität.

Business von Netzbetreibern: Stabilität des Netzes sicherstellen. Dabei
vom Vertrieb getrennt. Stromeinkauf einerseits langfristig (OTC),
andererseits kurzfristiger Zukauf von Ausgleichsenergie.

Vgl. auch Einleitung DuD (KJM)

TODO: Mit Bichler reden, um die Fakten richtig hinzubekommen.

Zentrale Frage: Wie kann man als Netzbetreiber mit volatiler Stromeinspeisung umgehen?
Traditioneller Weg: Zur Erzeugung von Ausgleichsstrom werden auch
Gaskraftwerke hoch- bzw. heruntergeregelt. Diese Kapazitäten werden in
Zukunf allerdings nicht ausreichen.

Im Prinzip zwei Möglichkeiten:
\begin{enumerate}
  \item Speicherung von Strom, z.B. in Pumpspeicherkraftwerken oder
	Batterien. Problem: Die Kapazitäten reichen nicht aus. 
  \item Demand-Side Management: Verlagerung von Lasten auf der
	Zeitachse.
\end{enumerate}

Schließlich führt die Erhebung von Stromverbrauchsinformationen auch zur
Veränderung von Konsumgewohnheiten. Stromkunden können, wenn
Informationen zum Momentanverbrauch unmittelbar zur Verfügung stehen,
ihr Verhalten direkt verändern und so ihren Strombezug um 15-20\%
reduzieren~\cite{geller2010smartgrid}. Ebenso ist es denkbar, mit den
gesammelten Strombezugsinformationen weiterführende Analysen
durchzuführen. Diese können zum Beispiel Verbraucher identifizieren, die
einen erheblichen Anteil am Stromverbrauch haben. Daraufhin können
automatisiert Hinweise gegeben werden, dass sich z.B. die Anschaffung
eines energiesparenden Kühlschranks schon nach einem Jahr amortisiert
hätte.


TODO: Grundlegende Terminologie festlegen.
Netzbetreiber, Verbraucher (gerät), Stromkunde, Strombezug (anstatt
Stromverbrauch)
